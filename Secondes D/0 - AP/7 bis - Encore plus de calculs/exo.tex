\documentclass{ap}

\renewcommand{\destinataire}{Secondes D}
\renewcommand{\numberdoc}{7}
\usepackage{tasks}
\settasks{label=\textbf{\alph*.}, label-format=\bfseries, label-offset=1em, label-align=right, before-skip =\smallskipamount, after-item-skip=0pt}

\title{Encore plus de calculs}

\begin{document}
% \maketitle

\vspace{-1em}

\looped{2}{
\begin{multicols}{2}
  \setcounter{exoscount}{3}
  \begin{exo}
\begin{enumerate}[nosep]
\item
  Résoudre les équations suivantes.
  \begin{enumerate}
      \item $(3x + 1)^2  = 4$
      \item  $2(x - 1)^2 + 16 = 0$
      \item  $-3(2x - 1)^2 - 27 = 0$
  \end{enumerate}

\item Pour $x\in \mathbb{R}$, soit $f(x) = 45 x^2 - 60 x + 45$. 

  \begin{enumerate}
  \item Montrer que $f(x) = 5(3 x - 2)^2 + 25$.
  \item En déduire les solutions de l'équation $f(x) = 0$.
  \end{enumerate}
\end{enumerate}

  \end{exo}

  
 \begin{exo}
\begin{enumerate}
\item \begin{enumerate}
\item Représenter à la calculatrice les fonctions :
  \begin{tasks}[label = \textbullet](2)
    \task $x \mapsto x^2$
    \task $x \mapsto (x - 1)^2$
    \task $x\mapsto (x - 2)^2$
    \task $x \mapsto (x + 3)^2$
  \end{tasks}
\item Soit $a\in \mathbb{R}$. Si $f$ et $g$ sont deux fonctions définies sur $\mathbb{R}$ par $f(x) = x^2$ et $g(x) = (x - a)^2$, expliquer le lien entre $\mathcal{C}_f$ et $\mathcal{C}_g$.
\end{enumerate}
\item \begin{enumerate}
\item Réprésenter à la calculatrice les fonctions~:
  \begin{tasks}[label = \textbullet](2)
    \task $x \mapsto x^2$
    \task $x \mapsto x^2 - 1$
    \task $x\mapsto x^2 - 2$
    \task $x \mapsto x^2 + 3$
  \end{tasks}

\item Soit $a\in \mathbb{R}$. Si $f$ et $g$ sont deux fonctions définies sur $\mathbb{R}$ par $f(x) = x^2$ et $g(x) = x^2 + a$, expliquer le lien entre $\mathcal{C}_f$ et $\mathcal{C}_g$.
  \end{enumerate}
\item Soient les fonctions définies sur $\mathbb{R}$ par :
\begin{tasks}(2)
  \task $f(x) = (x + 2)^2$
  \task $g(x) = x^2 + 2$
  \task* $h(x) = (x - 1)^2 + 3$
  \task* $k(x) = (x + 1)^2 - 3$
  \task* $u(x) = (x - 3)^2 + 1$
  \task* $v(x) = (x + 3)^2 - 1$
\end{tasks}

\includegraphics[width = \linewidth]{fs.png}
  
Associer à chaque fonction sa courbe représentative. 
  
  \end{enumerate}
 \end{exo} 
 \vspace{1em}
\end{multicols} 
}


\end{document}


