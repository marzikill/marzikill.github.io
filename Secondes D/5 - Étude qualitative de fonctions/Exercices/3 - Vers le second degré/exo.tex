\documentclass{exercice}

\renewcommand{\numberchap}{5}
\renewcommand{\destinataire}{Secondes D}
\renewcommand{\numberdoc}{3}
\usepackage{tasks}
\settasks{label=\textbf{\alph*.}, label-format=\bfseries, label-offset=1em, label-align=right, before-skip =\smallskipamount, after-item-skip=0pt}

\title{Vers le second degré}

\begin{document}
\maketitle

% - résolution d'inéquation : position relative de courbes
%     - bénéfices coûts
%     - sans contexte
% - résolution d'équation :
%     - avec des carrés, puis sous forme canonique
%     - application : résolution de l'équation de degré 2 (guidée).
% - forme canonique : tableau de variation

\vspace{-1cm}

\begin{multicols*}{2}
\begin{exo}
  \begin{enumerate}[nosep]
  \item Soit la fonction $f$ définie sur $\mathbb{R}$ par \\
    \centerline{$f(x) = -4x^2 + 14x - 12$}
    \begin{enumerate}
    \item Montrer que $f(x) = (2x - 3)(4 - 2x)$
    \item En déduire le tableau de signe de $f$.
    \end{enumerate}
  \item Soit la fonction $g$ définie sur $\mathbb{R}$ par \\
    \centerline{$g(x) = -x^2 + \dfrac{11}{2}x - 3$}
    \begin{enumerate}[nosep]
    \item Montrer que $g(x) = \left( \dfrac{1}{2}x + 3 \right) (1 - 2 x)$.
    \item En déduire le tableau de signe de $g$. 
    \end{enumerate}
  \end{enumerate}
\end{exo}

\begin{exo}
  Résoudre sur $\mathbb{R}$ les équations suivantes. 

  \begin{tasks}(2)
    \task $x^2 = 5$
    \task $x ^2 = \dfrac{4}{9}$
    \task $x^2 - 4 = 0$
    \task $x^2 + 3 = -5$
    \task $5x^2 = 125$
    \task $3x^2 - 75 = 0$
  \end{tasks}
\end{exo}

\begin{exo}
  Résoudre sur $\mathbb{R}$ les équations suivantes :

  \begin{enumerate}
    \item $2x^2 - 16 = 0$
    \item $2(x + 1)^2 - 16 = 0$
    \item $3(x - 1)^2 - 75 = 0$
    \item $-2(3x + 1)^2 + 32 = 0$
  \end{enumerate}
\end{exo}

\begin{exo}
\begin{enumerate}
\item Soit la fonction $f$ définie sur $\mathbb{R}$ par

  \centerline{$f(x) = 0,5(x + 1)^2 - 2$}

  \centerline{\includegraphics[width=0.8\linewidth]{"./f.png"}}




  \begin{enumerate}
  \item Répondre aux questions suivantes par lecture graphique.
    \begin{enumerate}[label=\textbf{\roman*.}]
    \item Dresser le tableau de signe de $f$.
    \item Dresser le tableau de variations de $f$.
    \end{enumerate}

  \item \begin{enumerate}[label=\textbf{\roman*.}]
    \item Montrer que $f(-1) = -2$
    \item Montrer que pour tout $x\in \mathbb{R}$,

      \centerline{$f(x) \geq f(-1)$}
    \item Commenter les réponses aux deux questions précédentes. 
    \end{enumerate}
  \end{enumerate}

\item Soit la fonction $g$ définie sur $\mathbb{R}$ par

  \centerline{$g(x) = 0,125(x - 2)^{2} + 1$}
  \begin{enumerate}
  \item À l'aide de votre calculatrice, conjecturer le tableau de variation de $g$.
  \item Démontrer que $g$ a pour minimum $1$, atteint en $2$.
  \end{enumerate}

\item Démontrer que la fonction $h$ définie sur $\mathbb{R}$ par $h(x) = -2(x + 5)^2 - 3$ a pour maximum $-3$, atteint en $-5$. 
\end{enumerate}
\end{exo}

\begin{exo}
    
        Une entreprise fabrique et vend des masques de protection. Elle peut produire au maximum 80 000 masques. Le coût de production (en milliers d'euros) de $q$ milliers de masques de protection est donné par :
            \[ 
                C(q) = 0,02q^2 + 0,1q + 9
            \]
            La recette, en milliers d'euros, engendrée par la vente de $q$ milliers de masques, est donnée par $R(q) = 1,2q$. 
        \begin{enumerate}
        \item \begin{enumerate}
            \item Quel est l'intervalle de définition des fonctions $C$ et $R$ ?
            \item Quel est le coût de fabrication d'un millier de masques de protection ?
            \item Combien est vendu un millier de masques de protection ?
            \item L'entreprise réalise-t-elle un bénéfice en produisant et vendant un millier de masques ?
          \end{enumerate}
        \item \begin{enumerate}
          \item À l'aide de votre calculatrice,  pour quelles valeurs de la production $q$ l'entreprise réalisera-t-elle un bénéfice ? Expliquer votre réponse. 
          \item On définit le bénéfice réalisé par l'entreprise lors de la vente de $q$ milliers de masques par $B(q) = R(q) - C(q)$.

            \noindent 
            Vérifier que : \\
                 \centerline{$B(q) = -0,02q^2 + 1,1q - 9$}
          \end{enumerate}
            \item \begin{enumerate}[wide=0pt, label=\textbf{\alph*.}]
                \item Démontrer que : \\
                    \centerline{$B(q) = (q - 10)(-0,02q + 0,9)$}
                  \item Dresser le tableau de signe de $B$ sur l'intervalle $[0 ; 80]$.
                \item Pour combien de milliers de masques de protections l'entreprise réalise-t-elle un bénéfice~?
            \end{enumerate}

          \item \begin{enumerate}
            \item Démontrer que

              \centerline{$B(q) = -0,02(x - 27,5)^2 + 6,125$}
            \item Conjecturer à l'aide de votre calculatrice le bénéfice maximal réalisé par l'entreprise.
            \item Démontrer cette conjecture. 
            \end{enumerate}
        \end{enumerate}
\end{exo}


\end{multicols*} 

\end{document}


